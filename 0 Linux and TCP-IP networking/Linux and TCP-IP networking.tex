\documentclass[10pt,a4paper]{article}
\usepackage{arabtex}
\usepackage[OT1,T1,LFE,LAE]{fontenc}
\usepackage[utf8]{inputenc}
\usepackage[arabic,english,farsi]{babel}
\usepackage{amsmath,amsfonts} % Math packages
\usepackage{amssymb}
%\usepackage{cmap}

\usepackage{multicol}

\usepackage{graphicx}
\usepackage[caption=false]{subfig}
\usepackage{color}
\usepackage{float}
\usepackage{sidecap}
%\sidecaptionvpos{figure}{c}
\usepackage{anysize}
\marginsize{2cm}{2cm}{2cm}{2cm}

\usepackage{listings}

\usepackage{appendix}
%\renewcommand{\appendixname}{Apéndices}
%\renewcommand{\appendixtocname}{Apéndices}
%\renewcommand{\appendixpagename}{Apéndices} 

\usepackage[colorlinks=true,plainpages=true,citecolor=blue,linkcolor=blue,urlcolor=cyan]{hyperref}
%\usepackage{hyperref}


%%% Equation and float numbering
\numberwithin{equation}{section}
\numberwithin{figure}{section}
\numberwithin{table}{section}


\newcommand{\horrule}[1]{\rule{\linewidth}{#1}}    % Horizontal rule

\newcommand{\titleText}{The Web, DHCP, NTP and NAT \\ Laboratory Manual}

\title{
\normalsize In the name of Allah\\
\vspace{10pt}
\LARGE\FR{بسم \allah الرحمن الرحیم}
\vspace{10pt}
\begin{center}
    %	\newcommand{\HRule}{\rule{\linewidth}{0.5mm}}
    \begin{minipage}{0.48\textwidth}
        \begin{flushleft}
            \includegraphics[height=64pt,width=64pt]{../img/logo.png}
        \end{flushleft}
    \end{minipage}
    \begin{minipage}{0.48\textwidth}
        \begin{flushright}
            \includegraphics[height=64pt]{../img/eng-logo.png}
        \end{flushright}
    \end{minipage}
\end{center}
\vspace*{-64pt}
%	\horrule{0.5pt} \\[0.4cm]
\huge \titleText\\
\vspace{40pt}
%	\horrule{2pt} \\[0.5cm]
}
\author{
\huge University of Tehran\\
\LARGE \FR{دانشگاه تهران}\\
\\
\LARGE School of Electrical and Computer Engineering\\
\FR{دانشکده مهندسی برق و کامپیوتر}\\
\\
\Large Computer Network Lab\\
\FR{آزمایشگاه شبکه‌های کامپیوتری}\\
\\
\\
\\
\normalfont
Dr. Ahmad Khonsari - \FR{احمد خونساری}\\
\href{mailto:a_khonsari@ut.ac.ir}{a\_khonsari@ut.ac.ir}\\
\\
\normalsize
Amir Haji Ali Khamseh'i - \FR{امیر حاجی علی خمسه‌ء}\\
\href{mailto:khamse@ut.ac.ir}{khamse@ut.ac.ir}\\
\\
\normalsize
Sina Kashi pazha - \FR{سینا کاشی پزها}\\
\href{mailto:sina\_kashipazha@ut.ac.ir}{sina\_kashipazha@ut.ac.ir}\\
\\
\normalsize
Amirahmad Khordadi - \FR{امیر احمد خردادی}\\
\href{mailto:a.a.khordadi@ut.ac.ir}{a.a.khordadi@ut.ac.ir}
}

\date{\vspace{30pt}\today\\\vspace{10pt}{\selectlanguage{farsi}\today}}

\usepackage{fancyhdr}
\pagestyle{fancy}
%\pagestyle{fancyplain}
\fancyhf{}
\fancyhead[L]{\footnotesize Computer Network Lab \\ \FR{آزمایشگاه شبکه‌های کامپیوتری}}
\fancyhead[R]{\footnotesize \titleText}
\fancyfoot[R]{\footnotesize School of Electrical and Computer Engineering\\\FR{دانشکده مهندسی برق و کامپیوتر}}
\fancyfoot[C]{\thepage}
\fancyfoot[L]{\footnotesize University of Tehran \\ \FR{دانشگاه تهران}}
\renewcommand{\footrulewidth}{0.8pt}
\renewcommand{\headrulewidth}{1pt}            % Remove header underlines
\renewcommand{\footrulewidth}{1pt}                % Remove footer underlines
\setlength{\headheight}{13.6pt}


\begin{document}
    \selectlanguage{english}

    %\vspace*{-1.5cm}
    \maketitle

    %\tableofcontents

    \pagebreak
    %Arabic \AR{\allah}
    %Persian \FR{فروشگاه}
    %Arabic \foreignlanguage{arabic}{كحض}
    %Persian \foreignlanguage{farsi}{فروشگاه}



    %	يعود تاريخ علوم الحاسوب إلى
    %\chapter
    %\section
    %\subsubsection
    %\subsection
    %\section
    %\chapter
    %\part
    %\caption
    %\end{otherlanguage}




    \section{Exercise 1}
    Run \textbf{ps -e} to list the processes running in \texttt{h1}.
    After starting a new process by running \textbf{telnet} in another command window, execute \textbf{ps -e} again in a third window to see if there is any change in its output. \\
    Find the process id of the \textbf{telnet} process you started, by: \\
    \centerline{\textbf{ps -e | grep telnet}} \\
    Then use \textbf{kill} \textit{process-id-of-telnet} to terminate the \textbf{telnet} process.
    \subsection*{Report}
    What is Internet service daemon (inetd)? \\
    Is \textbf{inetd} started in your system?
    Why? \\
    Is \textbf{xinetd} started in your system? What is its PID? \\

    %\pagebreak
    \section{Exercise 2}
    Display the file \texttt{/etc/services} on \texttt{h1} screen, using: \\
    \centerline{\textbf{more} \texttt{/etc/services}} \\
    Then in another console, use the redirect operator to redirect the \textbf{more} output to
    a file using \textbf{more} \texttt{/etc/services > ser-more}. Compare the file \texttt{ser-more} with the original \textbf{more} output in the other command window. \\
    Copy \texttt{/etc/services} file to a local file named \texttt{ser-cp} in your working directory,
    using \textbf{cp} \texttt{/etc/services ser-cp}. Compare files \texttt{ser-more} and \texttt{ser-cp}, using \textbf{cmp} \texttt{ser-more ser-cp}. Are these two files identical?\\
    Concatenate these two files using \textbf{cat} \texttt{ser-more ser-cp > ser-cat}. \\
    Display the file sizes using \textbf{ls -l} \texttt{ser*}. Save the output. What are the sizes of files \texttt{ser-more}, \texttt{ser-cp}, and \texttt{ser-cat}?

    %\pagebreak
    \section{Exercise 3}
    Read the \textbf{man} pages for the following programs:
    \begin{enumerate}
        \item arp
        \item arping
        \item ifconfig
        \item tcpdump
        \item ping
        \item netstat
        \item route
        \item wireshark
    \end{enumerate}
    Study the different options associated with each command.
    Throughout this lab you will use these commands rather extensively.
    \subsection*{Report}
    Explain the above commands briefly.
    Two or three sentences per command would be adequate.


    %\pagebreak
    \section{Exercise 4}
    In this exercise, we will use \textbf{tcpdump} to capture a packet containing the link, IP, and TCP headers and use ethereal to analyze this packet. \\
    First, run \textbf{tcpdump -enx -w} \texttt{dump.out} in \texttt{h1}.
    You will not see any \textbf{tcpdump} output, since the \textbf{-w} option is used to write the output to the \texttt{dump.out} file. \\
    Then, you may want to run \textbf{telnet} \texttt{10.0.0.2} to generate some TCP traffic.\footnote{Remember to run \textbf{\texttt{/etc/init.d/xinetd} restart} in \texttt{h2} to start telnet server on it.}
    After you login to \texttt{h2}, terminate the \textbf{telnet} session and terminate the \textbf{tcpdump} program.
    Next, you will use \textbf{wireshark} to open the packet trace captured by \textbf{tcpdump} and analyze the captured packets.
    To do this, run \textbf{wireshark} \texttt{dump.out \&}.
    The \textbf{wireshark} Graphical User Interface (GUI) will pop up and the packets captured by \textbf{tcpdump} will be displayed.
    Select any one of the packets that contain the link, IP, and TCP headers.
    \subsection*{Report}
    What is the value of the \texttt{protocol} field in the IP header of the packet you saved?
    What is the use of the \texttt{protocol} field? \\
    What is the value of the \texttt{frame type} field in an Ethernet frame carrying an IP datagram?


    \section{Exercise 5}
    This time we will run wireshark to capture an ARP request and an ARP reply in real-time. Simply run \textbf{wireshark \&} in \texttt{h1} and select the interface and start capturing.
    If there is no arp requests and replies in the network, generate some using \textbf{arping} \textit{10.0.0.2}. \\
    Now you should see several ARP replies in the arping output.
    \subsection*{Report}
    What is the value of the \texttt{frame type} field in an Ethernet frame carrying an ARP request and in an Ethernet frame carrying an ARP reply, respectively? \\
    What is the use of the \texttt{frame type} field? \\

    \section{Exercise 6}
    In \texttt{h1} run \textbf{wireshark \&} and select an interface to capture packets between hosts. \\
    Execute a TCP utility, telnet for example, in another command window: \\
    \textbf{telnet} \textit{10.0.0.2}
    \subsection*{Report}
    What are the port numbers used by the \texttt{h1} (local machine) and \texttt{h2} (remote machine)? \\
    Which machine’s port number matches the port number listed for \textbf{telnet} in the \texttt{/etc/services} file? \\

    \section{Exercise 7}
    In \texttt{h1} run \textbf{wireshark \&} and select an interface to capture packets between hosts. \\
    Then, \textbf{telnet} to the \texttt{h2} from a second command window by typing \textbf{telnet} \textit{10.0.0.2}.
    Again issue the same \textbf{telnet} \textit{10.0.0.2} command from a third command window.
    Now you are opening two \textbf{telnet} sessions to \texttt{h2} simultaneously, from two different command windows. \\
    Check the port numbers being used on both sides of the two connections from the output in the \textbf{wireshark} window. \\
    \subsection*{Report}
    When you have two \textbf{telnet} sessions with your machine, what port number is used on the \texttt{h2} (remote machine)? \\
    Are both sessions connected to the same port number on the \texttt{h2} (remote machine)? \\
    What port numbers are used in \texttt{h1} (local machine) for the first and second \textbf{telnet}, respectively? \\
    Explain briefly what a \texttt{socket} is. \\
\end{document}