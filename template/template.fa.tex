\documentclass{../UTNetLabFa}

\title{قالب و ساختار اسناد فارسی}

\begin{document}

\tableofcontents
\listoffigures
\listoftables
\lstlistoflistings
\pagebreak

\part{قالب}
\section{}
	\section{تصویر}
	برای درج تصویر لازم است تا دستور \LR{includegraphics} را در قسمت زبان انگلیسی قرار دهید.

    \begin{figure}[h]
    	\centering
        {\selectlanguage{english}\includegraphics[height=64pt]{../img/fanni}}
        \caption{\FR{تصویر دانشکده‌ی فنی - دانشگاه تهران}}
        \label{fig:fanni}
    \end{figure}

\subsection{فونت‌ها}
نمونه فونت‌های قابل استفاده:
\begin{table}[h]
	\label{tbl:farsi-fonts}
	\caption{جدول فونت های موجود}
	\centering
	\begin{tabular}{|l|l||R|R|}
		\hline
		\bfseries \LR{Font name}  & \bfseries\en{Command} & \bfseries نام & \bfseries    نمونه متن \\ \hline\hline
		\en{Nazli}      & \en{\textbackslash{}textnazli}  &      نازلی &  \textnazli{یک متن نمونه از فونت} \\ \hline
		\en{Nazli Bold} & \en{\textbackslash{}textnazlib} & نازلی درشت & \textnazlib{یک متن نمونه از فونت} \\ \hline
		\en{Titr Bold}  & \en{\textbackslash{}texttitr}   &       تیتر &   \texttitr{یک متن نمونه از فونت} \\ \hline
		\en{Homa}       & \en{\textbackslash{}texthoma}   &        هما &   \texthoma{یک متن نمونه از فونت} \\ \hline
	\end{tabular}
\end{table}

\begin{otherlanguage}{english}
	\section{English Section}
	This is a english section.%\footnote{\en{sample footnote}}
%	\subsection{English subsection}
	Subsection text.
%	\subsubsection{English SubSubSection}
	Subsubsection text.
\end{otherlanguage}

\end{document}
