\documentclass[10pt,a4paper]{article}
\usepackage{arabtex}
\usepackage[OT1,T1,LFE,LAE]{fontenc}
\usepackage[utf8]{inputenc}
\usepackage[arabic,english,farsi]{babel}
\usepackage{amsmath,amsfonts} % Math packages
\usepackage{amssymb}
%\usepackage{cmap}

\usepackage{multicol}

\usepackage{graphicx}
\usepackage[caption=false]{subfig}
\usepackage{color}
\usepackage{float}
\usepackage{sidecap}
\usepackage[normalem]{ulem}
%\sidecaptionvpos{figure}{c}
\usepackage{anysize}
\marginsize{2cm}{2cm}{2cm}{2cm}

\usepackage{listings}

\usepackage{appendix}
%\renewcommand{\appendixname}{Apéndices}
%\renewcommand{\appendixtocname}{Apéndices}
%\renewcommand{\appendixpagename}{Apéndices} 

\usepackage[colorlinks=true,plainpages=true,citecolor=blue,linkcolor=blue,urlcolor=cyan]{hyperref}
%\usepackage{hyperref}


%%% Equation and float numbering
\numberwithin{equation}{section}
\numberwithin{figure}{section}
\numberwithin{table}{section}


\newcommand{\horrule}[1]{\rule{\linewidth}{#1}}    % Horizontal rule

\newcommand{\titleText}{Network management and security \\ Laboratory Manual}

\title{
	\normalsize In the name of Allah\\
	\vspace{10pt}
	\LARGE\FR{بسم \allah الرحمن الرحیم}
	\vspace{10pt}
	\begin{center}
		%	\newcommand{\HRule}{\rule{\linewidth}{0.5mm}}
		\begin{minipage}{0.48\textwidth}
			\begin{flushleft}
				\includegraphics[height=64pt,width=64pt]{../img/logo.png}
			\end{flushleft}
		\end{minipage}
		\begin{minipage}{0.48\textwidth}
			\begin{flushright}
				\includegraphics[height=64pt]{../img/eng-logo.png}
			\end{flushright}
		\end{minipage}
	\end{center}
	\vspace*{-64pt}
	%	\horrule{0.5pt} \\[0.4cm]
	\huge \titleText\\
	\vspace{40pt}
	%	\horrule{2pt} \\[0.5cm]
}
\author{
	\huge University of Tehran\\
	\LARGE \FR{دانشگاه تهران}\\
	\\
	\LARGE School of Electrical and Computer Engineering\\
	\FR{دانشکده مهندسی برق و کامپیوتر}\\
	\\
	\Large Computer Network Lab\\
	\FR{آزمایشگاه شبکه‌های کامپیوتری}\\
	\\
	\\
	\\
	\normalfont
	Dr. Ahmad Khonsari - \FR{احمد خونساری}\\
	\href{mailto:a_khonsari@ut.ac.ir}{a\_khonsari@ut.ac.ir}\\
	\\
	\normalsize
	Amir Haji Ali Khamseh'i - \FR{امیر حاجی علی خمسه‌ء}\\
	\href{mailto:khamse@ut.ac.ir}{khamse@ut.ac.ir}\\
	\\
	\normalsize
	Sina Kashi pazha - \FR{سینا کاشی پزها}\\
	\href{mailto:sina\_kashipazha@ut.ac.ir}{sina\_kashipazha@ut.ac.ir}\\
	\\
	\normalsize
	Mohammad Ali Shahsavand - \FR{محمد علی شاهسوند}\\
	\href{mailto:mashahsavand@ut.ac.ir}{mashahsavand@ut.ac.ir}\\
	\\
	\normalsize
	Amirahmad Khordadi - \FR{امیر احمد خردادی}\\
	\href{mailto:a.a.khordadi@ut.ac.ir}{a.a.khordadi@ut.ac.ir}
}

\date{\vspace{30pt}\today\\\vspace{10pt}{\selectlanguage{farsi}\today}}

\usepackage{fancyhdr}
\pagestyle{fancy}
%\pagestyle{fancyplain}
\fancyhf{}
\fancyhead[L]{\footnotesize Computer Network Lab \\ \FR{آزمایشگاه شبکه‌های کامپیوتری}}
\fancyhead[R]{\footnotesize \titleText}
\fancyfoot[R]{\footnotesize School of Electrical and Computer Engineering\\\FR{دانشکده مهندسی برق و کامپیوتر}}
\fancyfoot[C]{\thepage}
\fancyfoot[L]{\footnotesize University of Tehran \\ \FR{دانشگاه تهران}}
\renewcommand{\footrulewidth}{0.8pt}
\renewcommand{\headrulewidth}{1pt}            % Remove header underlines
\renewcommand{\footrulewidth}{1pt}                % Remove footer underlines
\setlength{\headheight}{13.6pt}


\begin{document}
	\selectlanguage{english}
	
	%\vspace*{-1.5cm}
	\maketitle
	
	%\tableofcontents
	
	\pagebreak
	
	\section*{Exercises on secure applications}
	\section{Man in the Middle}
	In this excercise we will study security vulnerability of ftp and telnet protocol . To do so, wi will create mininet topology with single hub\footnote{hub forwards incoming packets to all of its ports, which means it always floods packets} connected to three hosts then we will connect from h1 to h2 through ftp and telnet connection and capture h1 password on h3. Let's do it.
	
	\begin{enumerate}
		\setlength{\itemindent}{10pt}
		\item Start pox controller with below command to force mininet switches act like hub.
		\begin{enumerate}
			\setlength{\itemindent}{10pt}
			\item [\$]  \texttt{python pox.py openflow.of\_01 {-}{-}address=127.0.0.1 {-}{-}port=6333 forwarding.hub}
		\end{enumerate}
		\item Run below command to start mininet with one single switch and three hosts and connect it to pox controller.
		\begin{enumerate}
			\setlength{\itemindent}{10pt}
			\item [\$]  \texttt{sudo mn {-}{-}topo single,3 {-}{-}controller remote,ip=127.0.0.1,port=6633}
		\end{enumerate}
		\item start ftp server on h2 with:
		\begin{enumerate}
			\setlength{\itemindent}{10pt}
			\item [h2>] \texttt{/usr/sbin/vsftpd}
		\end{enumerate}
		\item Run \textbf{\texttt{wireshark \&}} on h3
		\item Login to h2 and then run ftp from h1
		\begin{enumerate}
			\setlength{\itemindent}{10pt}
			\item [h1>] \texttt{ftp 10.0.0.2}
		\end{enumerate}
		\item Download 1 file (like configuration)
		\item Capture h1 password on wireshark output
		
	\end{enumerate}
	
	Repeat the above experiment, but use telnet to connect from h1 to h2 and capture h1 password on h3. \footnote{Don't foget to restart xinetd with \texttt{`/etc/init.d/xinetd restart`} on h2 to start telnet server. }
	
	\subsection*{Lab Report}
	\begin{enumerate}
		\setlength{\itemindent}{0pt}
		\item Can you see the login ID and the password in the \texttt{FTP} experiment? Submit the two packets you captured.
		\item Can you see the login ID and the password in the \texttt{TELNET} experiment? Submit the packets you captured.
		\item What is the difference between \texttt{FTP} and \texttt{TELNET} in their transmission of user ID’s and passwords? Which one is more secure?
	\end{enumerate}
	
	\section{Secure Transfer}
	Run previous mininet topology and connect it to pox controller but rather than using ftp and telnet use ssh and \texttt{sftp} as described in below steps.
	
	\begin{enumerate}
		\setlength{\itemindent}{10pt}
		\item Do step 1 and 2 frome previous section
		\item restart ssh service on h2 to enable \texttt{ssh} and \texttt{sftp} service on it with:
		\begin{enumerate}
			\setlength{\itemindent}{10pt}
			\item [h2>] \texttt{service ssh restart}
			\item [h2>] \texttt{/usr/lib/openssl/sftp-server \&}
		\end{enumerate}
		\item Run \texttt{wireshark \&} on h3
		\item Login to h2 \texttt{sftp} from h1 by:
		\begin{enumerate}
			\setlength{\itemindent}{10pt}
			\item [h1>] \texttt{sftp 10.0.0.2}
		\end{enumerate}
		\item Download 1 file (like configuration)
		\item Capture packets on wireshark output
		
	\end{enumerate}
	
	Repeat the above experiment, but use ssh and save the wireshark output for lab report.
	
	\subsection*{Lab Report}
	\begin{enumerate}
		\setlength{\itemindent}{0pt}
		\item In each experiment, can you extract the password from the tcpdump output? Can you read the \texttt{IP, TCP, SSH} headers? Can you read the \texttt{TCP} data?
		\item What is the client protocol (and version) used in both cases?
		\item What is the port number used by the \texttt{ssh} server? What is the port number used by the \texttt{sftp} server? Justify your answer using the wireshark output and the \texttt{/etc/services} file.
	\end{enumerate}
	
	
	\section*{Exercises on Firewalls and Iptables}
	\section{Firewall basic}
	Start mininet with default topology and Execute `\texttt{iptables -L -v}` on h1 and h2 to list the existing rules in the filter table. Save the output for the lab report.\\
	\\
	Append a rule to the end of the \texttt{INPUT} chain, by executing
	
	\begin{itemize}
		\setlength{\itemindent}{10pt}
		\item [h2>] \texttt{iptables -A INPUT -v -p TCP {-}{-}dport 23 -j DROP} 
	\end{itemize}
	
	\setlength{\parindent}{0pt}
	on h2. Run iptables -L -v again on both hosts to display the filter table. Save the output.\\
	
	
	\begin{enumerate}
		\setlength{\itemindent}{0pt}
		\item Start telnet server on h2 with \texttt{`/etc/init.d/xinetd restart`}.
		\item Capture packets on both hosts with wireshark
		\item Try to login with telnet from h1 to h2
	\end{enumerate}
	
	\subsection{Lab Report}
	\begin{enumerate}
		\setlength{\itemindent}{0pt}
		\item Can you telnet to the host from the remote machine?
		\item From the wireshark output, how many retries did \texttt{telnet} make? Explain the exponential \texttt{backoff} algorithm of \texttt{TCP} timeout and retransmission.
	\end{enumerate}
	
	\section{Firewall Action}
	Keep previous mininet running and delete the rule created in the last exercise on h2, by:
	
	\begin{itemize}
		\setlength{\itemindent}{10pt}
		\item [h2>] \texttt{iptables -D INPUT -v -p TCP {-}{-}dport 23 -j DROP}
	\end{itemize}
	
	\setlength{\parindent}{0pt}
	Then, append a new rule to the \texttt{INPUT} chain:
	
	\begin{itemize}
		\setlength{\itemindent}{10pt}
		\item [h2>] \texttt{iptables -A INPUT -v -p TCP {-}{-}dport 23 -j REJECT {-}{-}reject-with tcp-reset} 
	\end{itemize}
	
	\setlength{\parindent}{0pt}
	Execute \textbf{iptables -L -v} to display the new rule. On both machines in your topology, restart wireshark output, and then telnet from h1 to h2. Save the wireshark output for the lab report.
	
	\subsection{Lab Report}
	\begin{enumerate}
		\setlength{\itemindent}{0pt}
		\item Explain the difference between the wireshark outputs of this exercise and the previous exercise. How many attempts did \texttt{TCP} make this time?
	\end{enumerate}
	
	\pagebreak
	
	\section*{Exercises on secure Apache server}
	In the exercises in this section you don't need to create mininet topology, run command on your ubuntu terminal.
	\section{Raw HTTP}
	Create two files, \texttt{index.html} and \texttt{success.html}, in \texttt{/var/www/html} directory: \\
	\textbf{\$ sudo nano index.html}
	\begin{verbatim}
	<!-- index.html --> 
	<!DOCTYPE html>
	<html>
	<body>
	<form action="/success.html"  method="POST">
	Username:<br>
	<input type="text" name="user" value="user">
	<br>
	Password:<br>
	<input type="password" name="pass" value="1234">
	<br><br>
	<input type="submit" value="Submit">
	</form>
	</body>
	</html>
	\end{verbatim}
	\textbf{\$ sudo nano success.html}
	\begin{verbatim}
	<!-- success.html --> 
	<!DOCTYPE html>
	<html>
	<body>
	<h2>Goog Job!</h2>
	<a href="/">back to login form!</a>
	</body>
	</html>
	\end{verbatim}
	
	Now open your browser and enter your VM's IP address in URL bar and submit the login form:\\
	\texttt{http://172.18.133.XX} \\
	Save \textbf{wireshark} output for your lab report.
	\section{Secure HTTP}
	\subsection{Activate the SSL Module}
	Enable the module by typing: \\
	\textbf{\$ sudo a2enmod ssl} \\
	After you have enabled SSL, you'll have to restart the web server for the change to be recognized: \\
	\textbf{\$ sudo service apache2 restart} \\
	With that, our web server is now able to handle SSL if we configure it to do so. \\
	\subsection{Create a Self-Signed SSL Certificate}
	Let's start off by creating a subdirectory within Apache's configuration hierarchy to place the certificate files that we will be making:\\
	\textbf{\$ sudo mkdir /etc/apache2/ssl} \\
	Now that we have a location to place our key and certificate, we can create them both in one step by typing: \\
	\textbf{\$ sudo openssl req -x509 -nodes -days 365 -newkey rsa:2048 -keyout /etc/apache2/ssl/apache.key -out /etc/apache2/ssl/apache.crt}\footnote{You can use simple command as: `\texttt{make-ssl-cert generate-default-snakeoil --force-overwrite}' without editing default apache ssl config} \\
	Let's go over exactly what this means. \\
	\begin{itemize}
		\item \textbf{openssl}: This is the basic command line tool provided by OpenSSL to create and manage certificates, keys, signing requests, etc.
		\item \textbf{req}: This specifies a subcommand for X.509 certificate signing request (CSR) management. X.509 is a public key infrastructure standard that SSL adheres to for its key and certificate managment. Since we are wanting to create a new X.509 certificate, this is what we want.
		\item \textbf{-x509}: This option specifies that we want to make a self-signed certificate file instead of generating a certificate request.
		\item \textbf{-nodes}: This option tells OpenSSL that we do not wish to secure our key file with a passphrase. Having a password protected key file would get in the way of Apache starting automatically as we would have to enter the password every time the service restarts.
		\item \textbf{-days 365}: This specifies that the certificate we are creating will be valid for one year.
		\item \textbf{-newkey rsa:2048}: This option will create the certificate request and a new private key at the same time. This is necessary since we didn't create a private key in advance. The rsa:2048 tells OpenSSL to generate an RSA key that is 2048 bits long.
		\item \textbf{-keyout}: This parameter names the output file for the private key file that is being created.
		\item \textbf{-out}: This option names the output file for the certificate that we are generating.	
	\end{itemize}
	When you hit \texttt{ENTER}, you will be asked a number of questions. \\
	The questions portion looks something like this: \\
	
	\begin{verbatim}
	Country Name (2 letter code) [AU]:IR 
	State or Province Name (full name) [Some-State]:Tehran 
	Locality Name (eg, city) []:Tehran 
	Organization Name (eg, company) [Internet Widgits Pty Ltd]:University of Tehran 
	Organizational Unit Name (eg, section) []:ECE Department 
	Common Name (e.g. server FQDN or YOUR name) []:ece.ut.ac.ir 
	Email Address []:netlab@ut.ac.ir 
	\end{verbatim}
	The key and certificate will be created and placed in your \texttt{/etc/apache2/ssl} directory.
	\subsection{Configure Apache to Use SSL}
	Open the file with root privileges now:\\
	\textbf{\$ sudo nano /etc/apache2/sites-available/default-ssl.conf} \\
	You should make some changes to it. \\
	In the end, it will look something like this. The entries \texttt{SSLCertificateFile} and \texttt{SSLCertificateKeyFile}  were modified from the original file: \\
	\begin{verbatim}
	<IfModule mod_ssl.c>
	<VirtualHost _default_:443>
	ServerAdmin admin@example.com
	ServerName ece.ut.ac.ir 
	DocumentRoot /var/www/html
	...
	SSLCertificateFile /etc/apache2/ssl/apache.crt
	SSLCertificateKeyFile /etc/apache2/ssl/apache.key
	...
	</VirtualHost>
	</IfModule>
	\end{verbatim}
	Save and exit the file when you are finished.
	\subsection{Activate the SSL}
	Now that we have configured our SSL-enabled server, we need to enable it. \\
	We can do this by typing: \\
	\textbf{\$ sudo a2ensite default-ssl.conf} \\
	We then need to restart Apache to load our new file:\\
	\textbf{\$ sudo service apache2 restart} \\
	This should enable your server, which will serve encrypted content using the SSL certificate you created.
	\subsection{Test your Setup}
	Now that you have everything prepared, you can test your configuration by visiting your server's IP address after specifying the https:// protocol: \\
	\texttt{https://172.18.133.XX} \\
	Now submit the login form.
	Use \textbf{wireahark} output and examine the operation of SSL.
	
	\subsection*{Lab Report}
	\begin{enumerate}
		\setlength{\itemindent}{0pt}
		\item What is the port number used by the secure Apache server?
		\item Compare the general information of the received certificate with the make output saved in the last exercise. Are they consistent?
		\item What is the Subject of the received certificate? Who is the Issuer of this certificate? Are they the same?
		\item What is the Certificate Signature Algorithm used to generate and distribute this certificate?
		\item When was the certificate signed? When will it expire?
	\end{enumerate}
	
\end{document}