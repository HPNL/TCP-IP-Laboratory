\documentclass{../UTNetLabFa}

\title{کلیات شبکه و پشته‌ی \textLR{TCP-IP}}


\begin{document}

\tableofcontents
\listoffigures
\listoftables
\lstlistoflistings
%\listofcodes
\pagebreak

\part{تعاریف}
\section{شبکه}
    \section{اینترنت}
    اینترنت یک سیستم جهانی است که از اتصال میلیون ها کامپیوتر در سراسر جهان ساخته شده است.

    برای تنظیم کردن مسیریاب از دستورات زیر استفاده کنید:

    \begin{figure}[h]
    	\centering
        {\selectlanguage{english}\includegraphics[height=64pt]{../img/fanni}}
        \caption{\FR{تصویر دانشکده‌ی فنی - دانشگاه تهران}}
        \label{fig:fanni}
    \end{figure}

\subsection{فونت‌ها}
نمونه فونت‌های قابل استفاده:
\begin{table}[h]
	\label{tbl:farsi-fonts}
	\caption{جدول فونت های موجود}
	\centering
	\begin{tabular}{|l|l||R|R|}
		\hline
		\bfseries \LR{Font name}  & \bfseries\en{Command} & \bfseries نام & \bfseries    نمونه متن \\ \hline\hline
		\en{Nazli}      & \en{\textbackslash{}textnazli}  &      نازلی &  \textnazli{یک متن نمونه از فونت} \\ \hline
		\en{Nazli Bold} & \en{\textbackslash{}textnazlib} & نازلی درشت & \textnazlib{یک متن نمونه از فونت} \\ \hline
		\en{Titr Bold}  & \en{\textbackslash{}texttitr}   &       تیتر &   \texttitr{یک متن نمونه از فونت} \\ \hline
		\en{Homa}       & \en{\textbackslash{}texthoma}   &        هما &   \texthoma{یک متن نمونه از فونت} \\ \hline
	\end{tabular}
\end{table}



\begin{otherlanguage}{english}
	\section{English Section}
	This is a english section.%\footnote{\en{sample footnote}}
%	\subsection{English subsection}
	Subsection text.
%	\subsubsection{English SubSubSection}
	Subsubsection text.
\end{otherlanguage}

\part{ادوات شبکه}
\section{کابل‌ها}
در این بخش توضیحاتی در رابطه با ادوات شبکه قرار می‌گیرد.
\subsection{سیمی}
انواع ارتباطات سیمی

\subsection{بی سیم}
	اتصالاتی همانند \textLR{WiFi} و \textLR{Bluetooth}.
\subsection{نوری}
    اتصالات فیبر نوری به علت نویز کمتر در مسیر، امکان ارسال به مسیرهای دور را با سرعت بالا و خطای کم را فراهم می‌کنند.
\begin{question}
    \item متن سوال
\end{question}

\begin{report}
	\item نمونه گزارش
\end{report}

\section{دستگاه‌ها}

\subsection{هاب}
\subsection{سوئیچ}
\subsection{مسیریاب}
\subsubsection{تنظیمات مسیریاب}
{
	\selectlanguage{english}
\begin{lstlisting}[language=cisco,caption=\FR{تنظیمات مسیریاب}]
config term
	no ip routing
	bridge 1 protocol ieee ! for STP protocol
	int f0/0
		ip addr 128.238.61.1 255.255.255.0
		bridge-group 1
		no shut
		exit
	int f0/1
		ip addr 128.238.61.2 255.255.255.0
		bridge-group 1
		no shut
		end
	Ctrl+Z
\end{lstlisting}
}
\subsection{سوئیچ های نسل جدید}
\subsubsection{کنترل کننده}
\subsection{پایشگر (\en{FireWall})}

\part{شبیه سازی}


\end{document}
