\documentclass{../UTNetLabFa}

\title{کلیات شبکه و پشته‌ی \textLR{TCP-IP}}

\begin{document}	
%	\tableofcontents
%	\newpage

\part{ابتدایی}
    \section{اینترنت}
    اینترنت یک سیستم جهانی است که از اتصال میلیون ها کامپیوتر در سراسر جهان ساخته شده است.
\subsection{تنظیمات مسیریاب}
    برای تنظیم کردن مسیریاب از دستورات زیر استفاده کنید:

    \begin{figure}[h]
    	\centering
        {\selectlanguage{english}\includegraphics[height=64pt]{../img/fanni}}
    \end{figure}

{
    \selectlanguage{english}
    Router configuration:
        \begin{lstlisting}[language={cisco}]
config term
    no ip routing
    bridge 1 protocol ieee # for STP protocol
    int f0/0
        ip addr 128.238.61.1 255.255.255.0
        bridge-group 1
        no shut
        exit
    int f0/1
        ip addr 128.238.61.2 255.255.255.0
        bridge-group 1
        no shut
        end
    Ctrl+Z
    \end{lstlisting}
}

\begin{otherlanguage}{english}
	\section{English Section}
	This is a english section.
\end{otherlanguage}

\section{ادوات شبکه}
در این بخش توضیحاتی در رابطه با ادوات شبکه قرار می‌گیرد.
\subsection{سیمی}
انواع ارتباطات سیمی

\subsection{بی سیم}
	اتصلاتی همانند \textLR{WiFi} و \textLR{Bluetooth}.
\subsection{نوری}
\begin{report}
\item نمونه گزارش
\end{report}

\end{document}
