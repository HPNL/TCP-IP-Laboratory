% !TeX spellcheck = en_US

\documentclass[aspectratio=169,15pt]{beamer}

\usepackage[utf8]{inputenc}
\usepackage{xcolor}
\usepackage{hyperref}
\usepackage{makebox}
\usepackage{textpos}

\hypersetup{colorlinks = true}

\usetheme{Madrid}
% \usetheme{Szeged}
\usecolortheme{beaver}

\title{Computer Networks Laboratory}
\subtitle{Chapter 0: Introduction}
\author{Ahmad Khonsari}
\institute[ECE @ UT]{
    Electrical and Computer Engineering Department\\
    Tehran University \newline
    \url{http://ece.ut.ac.ir/}
}
\date{Spring 1400}
\subject{Computer Science}

% \logo{
%     \includegraphics[height=1.4cm]{../img/ut.pdf}
% }
\titlegraphic{
    \makebox[0.9\paperwidth]{
        \includegraphics[height=1.4cm]{../img/fanni.png}
        \hfill
        \includegraphics[height=1.4cm]{../img/ut.pdf}
    }
}

\setbeamercolor{section number projected}{bg=red,fg=white}
\setbeamercolor*{item}{fg=red}

\beamertemplatenavigationsymbolsempty

\begin{document}

\frame{\titlepage}

\addtobeamertemplate{frametitle}{}{%
    \begin{textblock*}{100mm}(.85\textwidth,-1cm)
        \includegraphics[height=1cm,width=1.33cm]{../img/fanni.png}
        \includegraphics[height=1cm,width=1cm]{../img/ut.pdf}
    \end{textblock*}
}

\begin{frame}
    \frametitle{Table of Contents}
    \tableofcontents
\end{frame}

\section{Teaching Staff}
\begin{frame}
    \frametitle{Teaching Staff}
    \begin{itemize}
        \item Instructor:
              \begin{itemize}
                  \item Ahmad Khonsari: \alert{a\_khonsari @ut .ac .ir}
              \end{itemize}
        \item Chief Teaching assistants:
              \begin{itemize}
                  \item Amir Haji Ali Khamseh’i: \alert{khamse @ut .ac .ir}
                  \item Reza Sharifi nia: \alert{reza.sharifnia @ut .ac .ir}
              \end{itemize}
        \item Contact time: Tuesday-Wednesday (netlab-softlab (Azmayeshagah Narmafzar))
    \end{itemize}

\end{frame}

\section{Course Goals}
\begin{frame}
    \frametitle{Course Goals}

    Learn:
    \begin{itemize}
        \item How to configure hosts \& networking devices
        \item How to connect network devices
        \item How to simulate network
        \item Network protocols such as: SNMP, Routing protocols, ARP, RIP, OSPF, UDP, TCP, Multicast, NAT, HTTP, DHCP, SNMP, Security, \dots
    \end{itemize}

    \begin{alertblock}{Note}
        Some materials of the lab has not been covered in your "Computer Network Course" and you need to study additional references
    \end{alertblock}

\end{frame}

\section{Course Background}
\begin{frame}
    \frametitle{Course Background}

    \begin{itemize}
        \item Computer Net Lab is a one unit course and it is a follow up of the Computer Networks course.
        \item The materials you have learned in the computer network course is necessary but not sufficient to do Computer Net Lab.
              You will learn at least 40\% of networking materials that you didn’t see in the Computer Networks course (such as IPV6, Spanning Tree, Multicast, MPLS).
              It is necessary to learn the necessary background before attending each session.
        \item The grading of this course is divided to 2 main parts.
              \begin{enumerate}
                  \item Reading the background material of each lab before attending each session.
                        There is a quiz in the beginning of each lab of the main materials of the current session.
                  \item Carrying out the lab within the time slot that assigned for each lab and preparing the report.
              \end{enumerate}
    \end{itemize}

\end{frame}

\section{Course Format}
\begin{frame}
    \frametitle{Course Format}

    \begin{itemize}
        \item 10 sessions
        \item Covers the materials presented in the book:
              \begin{itemize}
                  \item Shivendra S. Panwar,“TCP/IP essential a lab based approach”, Cambridge University Press, 2004
                  \item[+] One optional new Lab
              \end{itemize}
        \item 10 lab reports + TA quizzes
    \end{itemize}
    \begin{alertblock}{Note:}
        Delay is not allowed: please enter each session on time.
    \end{alertblock}

\end{frame}

\section{Class Program}
\begin{frame}
    \frametitle{Class Program}

    \begin{enumerate}
        \item \alert{10 minutes} Quiz in the beginning of each session. You have to read the corresponding part of the book.
        \item \alert{10 minutes} TA explanation (or watch the recorded videos)\footnote{Maybe you need to watch the video before class to save time to do lab instructions}
        \item \alert{2 hours and 30 minutes} Fulfill the experiments according to lab instructions
        \item Fulfill and submit your lab report
              \begin{itemize}
                  \item We use Moodle Quiz platform to collect your reports
              \end{itemize}
    \end{enumerate}
    \begin{alertblock}{Note:}
        You need to submit your report  within the assigned time slot of each lab.
        (If you didn’t finish your lab completely within the assigned time slot, you need to submit the incomplete report.)
    \end{alertblock}

\end{frame}

\section{Class rules}
\begin{frame}
    \frametitle{Class rules}

    LATE ARRIVAL:
    {
    \setbeamercolor{block body}{fg=black, bg=green!10!white}
    \setbeamercolor{block title}{fg=white, bg=green!40!black}
    \begin{block}{Below 15 minutes}
        Don’t worry (but you may miss the quiz)
    \end{block}
    }
    {
    \setbeamercolor{block body}{fg=black, bg=orange!10!white}
    \setbeamercolor{block title}{fg=white, bg=orange!50!black}
    \begin{block}{Between 15 and 30 minutes}
        Get \alert{-0.5} penalty score for your session grade
    \end{block}
    }
    \begin{alertblock}{More than 30 minutes}
        Don’t enter the class. you miss \alert{2} points of this session
    \end{alertblock}

\end{frame}

\section{Grading}
\begin{frame}
    \frametitle{Grading}

    \begin{itemize}
        \item 10 labs: score of each lab is \alert{2} (sums up to a total of \alert{20} for 10 labs)
        \item For remote class at time of \alert{COVID-19} outbreak
              \begin{itemize}
                  \item {\color{red} 0.4} to {\color{red} 0.5} for each quiz (0.1 Additional Score)
                  \item {\color{red} 1.6} for each lab report
              \end{itemize}
        \item For normal class
              \begin{itemize}
                  \item {\color{red} 0.2} to {\color{red} 0.3} for each quiz (0.1 Additional Score)
                  \item {\color{red} 1.4} for finishing up the lab
                  \item {\color{red} 0.4} for each lab report (1 week)
              \end{itemize}
    \end{itemize}
    \begin{alertblock}{Note:}
        You can only object your grades within {\color{red} 2 days} of receiving them. We do not consider objections after that.
        Please contact instructor of the course less than {\color{red} 1 day} if your objection remains unresolved within this time.
    \end{alertblock}

\end{frame}

\section{Course Materials}
\begin{frame}
    \frametitle{Course Materials}

    \begin{itemize}
        \item Contents of the lab are available at:\\
              \url{https://github.com/UT-Network-Lab/TCP-IP-Laboratory}
        \item GNS3 on Linux (physical or virtual machine)
              \begin{itemize}
                  \item \href{https://github.com/UT-Network-Lab/TCP-IP-Laboratory/blob/master/README.md}{README.md} of Installation instructions
                  \item \href{https://www.dropbox.com/s/bad8eongryfnylr/GNS3\%20Installation\%20Tutorial.mp4?dl=0}{Video} of Installation instructions
              \end{itemize}
        \item Docker images (\href{https://hub.docker.com/r/utnetlab/term}{netlab/term}, \href{https://hub.docker.com/r/utnetlab/gui}{netlab/gui}, \href{https://github.com/orgs/UT-Network-Lab/packages?repo_name=docker-tools}{Github-Packages})
        \item Laboratory instructions (\href{https://github.com/UT-Network-Lab/TCP-IP-Laboratory/releases/latest}{Documents})
        \item GNS3 Portable projects (\href{https://github.com/UT-Network-Lab/gns3-figures/releases/latest}{Figures})
        \item GNS3 simple \href{https://github.com/UT-Network-Lab/TCP-IP-Laboratory/blob/master/gns3.md}{manual}
    \end{itemize}

\end{frame}

\begin{frame}
    \frametitle{Consider these rules in your report answers}

    \begin{itemize}
        \item Don’t submit screen shot \textbf{images}, except for report questions that specifically ask for an image.
              \begin{itemize}
                  \item Moodle has a bug for save and submit images (please contact TAs).
                  \item Your answers should be short (concise and useful).
              \end{itemize}
        \item Provide your opinion or analysis to answer the questions, in your own word. Do not copy and paste a text or document as your answers.
        \item Your answer should be in accordance with the resources and definitions (You need to read the resources before the session).
        \item You need to choose the best answer in multiple choice questions.
        \item Answer each question separately. Do not refer to answers of previous questions  (you miss the grades of incomplete answers).
              % upload numerical results of this question in your report
    \end{itemize}

\end{frame}

\end{document}