\documentclass{../UTNetLab}

\title{TCP and Its Applications}
\newcommand\reference{
   S. Panwar, S. Mao, J.-dong Ryoo, and Y. Li, “TCP study,” in TCP/IP Essentials: A Lab-Based Approach, Cambridge: Cambridge University Press, 2004, pp. 111–133.
}

\begin{document}
\part{Exercises on TCP Connection Control}
    Like previous lab, connect two host with one hub together (Figure~5.0) or use Figure~1.3 with two host.

\section{Telnet terminal}
    While \lstinline[emph={h0,h1,netlab}]{tcpdump -S h0.netlab and h1.netlab} is running, execute: \lstinline[emph={h0,h1,netlab}]{telnet h1.netlab echo} on the \textit{h0} host, then type some text.
    Save the \lstinline{tcpdump} output.
    
    \begin{report}
        \item Explain TCP connection establishment and termination using the \lstinline{tcpdump} output.
        
        \item What were the announced MSS values for the two hosts?
        
        \item What happens if there is an intermediate network that has an MTU less than the MSS of each host?
    See if the \texttt{DF}\footnote{Don't Fragment} flag was set in \lstinline{tcpdump} output.
    You can change interface \texttt{MTU} with \lstinline[emph=$eth]{sudo ifconfig $eth mtu 1400}
    \end{report}
    
\section{TCP vs UDP Connection Establishment}
    While \lstinline[emph={h0,h1,netlab}]{tcpdump -nx h0.netlab and h1.netlab} is running, use \lstinline{socket}\footnote{Basic command is \lstinline{sock}. Use alternative \lstinline{socket} (linked to \lstinline{sock}).} to send a UDP datagram to the \textit{h1} host from the \textit{h0} machine:
    \begin{lstlisting}[emph={h0,h1,netlab}]
socket -u -i -n1 h1.netlab 8888
    \end{lstlisting}
    {Save} the \lstinline{tcpdump} or \lstinline{wireshark} output for your lab report. \\
    Restart the above \lstinline{tcpdump} command, execute \lstinline{socket} in the TCP mode:
    \begin{lstlisting}[emph={h1,netlab}]
socket -i -n1 h1.netlab 8888
    \end{lstlisting}
    {Save} the \lstinline{tcpdump} output for your lab report.
    
    \begin{report}
        \item Explain what happened in both the UDP and TCP cases.
                When a client requests a non-existing server, how do UDP and TCP handle this request, respectively?
    \end{report}

\part{Exercise on TCP Interactive Data Flow}
\section{Interactive Data Flow}
While \lstinline{tcpdump} is capturing the traffic between \textit{h0} machine and a \textit{h1} machine, issue following commands:
    \begin{lstlisting}[emph={h1,netlab}]
tcpdump -nv # or run wireshark
telnet h1.netlab
    \end{lstlisting}
    After logging in to the host, type \lstinline{date} and press the {Enter} key.

    Now, in order to generate data faster than the round-trip time of a single byte to be sent and echoed, type any sequence of keys in the \lstinline{telnet} window very rapidly.\footnote{For example hold ``A'' key or write ``qwertyuiop'' in \lstinline{telnet} window.}

    {Save} the \lstinline{tcpdump} output for your lab report.
    
    \begin{report}[{Answer the following questions, based upon the \lstinline{tcpdump} output saved in the above exercise.}]
        \item What is a delayed acknowledgement?
            What is it used for?
        
        \item Can you see any delayed acknowledgements in your \lstinline{tcpdump} output?

            If yes, explain the reason.
            Mark some of the lines with delayed acknowledgements, and submit the \lstinline{tcpdump} output with your report.

            Explain how the delayed ACK timer operates from your \lstinline{tcpdump} output.

            If you don’t see any delayed acknowledgements, explain the reason why none was observed.
        
        \item What is the \textit{Nagle}\footnote{Nagle Algorithm is a means of improving the efficiency of TCP/IP networks by reducing the number of packets that need to be sent over the network.} algorithm used for?

            From your \lstinline{tcpdump} output, can you tell whether the \textit{Nagle} algorithm is enabled or not? Give the reason for your answer.

            From your \lstinline{tcpdump} output for when you typed very rapidly\footnote{You can paste a large string into terminal, about 50-100 byte.}, can you see any segment that contains more than one character going from the \textit{h0} host to the \textit{h1} machine?
    \end{report}

\part{Exercise on TCP Bulk Data Flow}
\section{IP Segment}
    While \lstinline{tcpdump} is running and capturing the packets between \textit{h0} machine and a \textit{h1} machine, on the \textit{h1} machine, which acts as the server, execute:
    \begin{lstlisting}
socket -i -s 7777
    \end{lstlisting}
    Then, on the \textit{h0} machine, which acts as the client, execute:
    \begin{lstlisting}[emph={h1,netlab}]
socket -i -n16 h1.netlab 7777
    \end{lstlisting}
    Do the same experiment three times.

    Save all the \lstinline{tcpdump} outputs for your lab report.
    
    \begin{report}
        \item Using one of three \lstinline{tcpdump} outputs, explain the operation of TCP in terms of data segments and their acknowledgements.
                Does the number of data segments differ from that of their acknowledgements?

            Compare all the \lstinline{tcpdump} outputs you saved.
            Discuss any differences among them, in terms of data segments and their acknowledgements.
        
        \item From the \lstinline{tcpdump} output, how many different TCP flags can you see? Enumerate the flags and explain their meanings.

            How many different TCP options can you see?
            Explain their meanings.
    \end{report}

\part{Exercises on TCP Timers and Retransmission}
\section{Keepalive parameter}
    Execute \lstinline{sysctl -A | grep keepalive} to display the default values of the TCP kernel parameters that are related to the TCP keepalive timer.

    \begin{report}
        \item What is the default value of the TCP keepalive timer?
        
        \item What is the maximum number of TCP keepalive probes a host can send?
    \end{report}

\section{TCP Retransmission}
    While \lstinline{tcpdump} is running to capture the packets between the \textit{h0} host and the \textit{h1} host, start a \lstinline{socket} server on the \textit{h1} host,
    \begin{lstlisting}
socket -s 8888
    \end{lstlisting}
    Then, execute the following command on the \textit{h0} host,
    \begin{itemize}
        \item add link delay in simulator\footnote{In GNS3, \lstinline{right click} on link, select \lstinline{Packet Filter} and set link \lstinline{delay} to \textit{100 ms}}
        \item \lstinline[emph={host}]{socket -i -n200 -p 100 host 8888}
    \end{itemize}
    While the sender is injecting data segments into the network, shutdown the network interface on the \textit{h1} host that connect the sender to the hub for about ten seconds.
    \begin{lstlisting}[emph={eth0}]
# use one of below commands
ip link set eth0 down
ifconfig eth0 down
    \end{lstlisting}

    After observing several retransmissions, set network interface up:
after seconds\ldots
    \begin{lstlisting}[emph={eth0}]
# use one of below commands
ip link set eth0 up
ifconfig eth0 up
    \end{lstlisting} %in order to connect again%
    When all the data segments are sent, save the \lstinline{tcpdump} output for the lab report.
    
    \begin{report}
        \item Submit the \lstinline{tcpdump} output saved in this exercise.
        
        \item From the \lstinline{tcpdump} output, identify when the cable was disconnected.
        
        \item Describe how the retransmission timer changes after sending each retransmitted packet, during the period when the cable was disconnected.
        
        \item Explain how the number of data segments that the sender transmits at once (before getting an ACK) changes after the connection is reestablished.\footnote{Can see TCP \textbf{window scale} option and \textbf{RTT}}.
    \end{report}
    
\part{Other Exercises}
\section{Fragmentation}
    While \lstinline[emph={h0,h1,netlab}]{tcpdump src host h0.netlab} is running, execute the following command, which is similar to the command we used to find out the maximum size of a UDP datagram in previous lab session (Chapter~5 of reference book),
    \begin{lstlisting}[emph={n, h1,netlab}]
socket -i -n1 -w n h1.netlab echo
    \end{lstlisting}
    Let \textit{n} be larger than the maximum UDP datagram size we found in previous lab session.

    As an example, you may use $n = 70080$.

    \begin{report}
        \item Did you observe any IP fragmentation?
            
        \item If IP fragmentation did not occur this time, how do you explain this compared to what you observed in previous lab session for UDP packets?
    \end{report}

\section{Linux TCP/IP Kernel Parameter}
    Study the manual page of \lstinline{/sbin/sysctl}.
    Examine the default values of some TCP/IP configuration parameters that you might be interested in.
    Examine the configuration files in the \path{/proc/sys/net/ipv4} directory.

    \begin{report}
        \item Explain what is \lstinline{sysctl} command for?
            
        \item Explain two arbitrary TCP/IP configuration parameters.
                What is their default values?
            
        \item Name two arbitrary file in the \path{/proc/sys/net/ipv4} directory.
                What is their content?
    \end{report}

\end{document}