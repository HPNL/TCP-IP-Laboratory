\documentclass[10pt,a4paper]{article}
\usepackage{arabtex}
\usepackage[OT1,T1,LFE,LAE]{fontenc}
\usepackage[utf8]{inputenc}
\usepackage[arabic,english,farsi]{babel}
\usepackage{amsmath,amsfonts} % Math packages
\usepackage{amssymb}
%\usepackage{cmap}

\usepackage{multicol}

\usepackage{graphicx}
\usepackage[caption=false]{subfig}
\usepackage{color}
\usepackage{float}
\usepackage{sidecap}
%\sidecaptionvpos{figure}{c}
\usepackage{anysize}
\marginsize{2cm}{2cm}{2cm}{2cm}

\usepackage{listings}

\usepackage{appendix}
%\renewcommand{\appendixname}{Apéndices}
%\renewcommand{\appendixtocname}{Apéndices}
%\renewcommand{\appendixpagename}{Apéndices} 

\usepackage[colorlinks=true,plainpages=true,citecolor=blue,linkcolor=blue,urlcolor=cyan]{hyperref}
%\usepackage{hyperref}


%%% Equation and float numbering
\numberwithin{equation}{section}
\numberwithin{figure}{section}
\numberwithin{table}{section}


\newcommand{\horrule}[1]{\rule{\linewidth}{#1}}    % Horizontal rule

\newcommand{\titleText}{OSPF and Socket Programming\\ Laboratory Manual}

\title{
\normalsize In the name of Allah\\
\vspace{10pt}
\LARGE\FR{بسم \allah الرحمن الرحیم}
\vspace{10pt}
\begin{center}
    %	\newcommand{\HRule}{\rule{\linewidth}{0.5mm}}
    \begin{minipage}{0.48\textwidth}
        \begin{flushleft}
            \includegraphics[height=64pt,width=64pt]{../img/logo.png}
        \end{flushleft}
    \end{minipage}
    \begin{minipage}{0.48\textwidth}
        \begin{flushright}
            \includegraphics[height=64pt]{../img/eng-logo.png}
        \end{flushright}
    \end{minipage}
\end{center}
\vspace*{-64pt}
%	\horrule{0.5pt} \\[0.4cm]
\huge \titleText\\
\vspace{40pt}
%	\horrule{2pt} \\[0.5cm]
}
\author{
\huge University of Tehran\\
\LARGE \FR{دانشگاه تهران}\\
\\
\LARGE School of Electrical and Computer Engineering\\
\FR{دانشکده مهندسی برق و کامپیوتر}\\
\\
\Large Computer Network Lab\\
\FR{آزمایشگاه شبکه‌های کامپیوتری}\\
\\
\\
\\
\normalfont
Dr. Ahmad Khonsari - \FR{احمد خونساری}\\
\href{mailto:a_khonsari@ut.ac.ir}{a\_khonsari@ut.ac.ir}\\
\\
\normalsize
Amir Haji Ali Khamseh'i - \FR{امیر حاجی علی خمسه‌ء}\\
\href{mailto:khamse@ut.ac.ir}{khamse@ut.ac.ir}\\
\\
\normalsize
Sina Kashi pazha - \FR{سینا کاشی پزها}\\
\href{mailto:sina\_kashipazha@ut.ac.ir}{sina\_kashipazha@ut.ac.ir}\\
\\
\normalsize
Mohammad Ali Shahsavand - \FR{محمد علی شاهسوند}\\
\href{mailto:mashahsavand@ut.ac.ir}{mashahsavand@ut.ac.ir}\\
\\
\normalsize
Amirahmad Khordadi - \FR{امیر احمد خردادی}\\
\href{mailto:a.a.khordadi@ut.ac.ir}{a.a.khordadi@ut.ac.ir}
}

\date{\vspace{30pt}\today\\\vspace{10pt}{\selectlanguage{farsi}\today}}

\usepackage{fancyhdr}
\pagestyle{fancy}
%\pagestyle{fancyplain}
\fancyhf{}
\fancyhead[L]{\footnotesize Computer Network Lab \\ \FR{آزمایشگاه شبکه‌های کامپیوتری}}
\fancyhead[R]{\footnotesize \titleText}
\fancyfoot[R]{\footnotesize School of Electrical and Computer Engineering\\\FR{دانشکده مهندسی برق و کامپیوتر}}
\fancyfoot[C]{\thepage}
\fancyfoot[L]{\footnotesize University of Tehran \\ \FR{دانشگاه تهران}}
\renewcommand{\footrulewidth}{0.8pt}
\renewcommand{\headrulewidth}{1pt}            % Remove header underlines
\renewcommand{\footrulewidth}{1pt}                % Remove footer underlines
\setlength{\headheight}{13.6pt}


\begin{document}
    \selectlanguage{english}

    %\vspace*{-1.5cm}
    \maketitle

    %\tableofcontents

    \pagebreak
    %Arabic \AR{\allah}
    %Persian \FR{فروشگاه}
    %Arabic \foreignlanguage{arabic}{كحض}
    %Persian \foreignlanguage{farsi}{فروشگاه}



    %\chapter
    %\section
    %\subsubsection
    %\subsection
    %\section
    %\chapter
    %\part
    %\caption
    %\end{otherlanguage}

    
    \section*{Socket programming exercises 1}
    Examine the UDP socket programs \texttt{/home/guest/UDPserver.c} and \texttt{/home/guest/UDPclient.c} to learn how to write a UDP socket program.
    Compile the C programs using \texttt{gcc -o UDPserver UDPserver.c -lnsl} and \texttt{gcc -o UDPclient UDPclient.c -lnsl}. \\
    Start \textbf{wireshark} to capture packets from or to a remote host. \\
    On the remote host, start the UDP server by \textbf{UDPserver server\_port}.
    Then, start the UDP client on your host by \textbf{UDPclient remote\_host server\_port a\_message}.
    You may execute the UDP client program on other hosts to connect to the same UDP server.
    Terminate \textbf{wireshark}, examine its output and compare the output with the UDP server and client outputs.
    Repeat the above experiments, but now use the \texttt{TCPserver.c} and \texttt{TCPclient.c}.

    \section*{Socket programming exercises 2}
    Execute \textbf{man setsockopt} to display the various socket options and how to set them.
    Examine the \textbf{netspy} and \textbf{netspyd} source code in Appendix C.2 to see how to create a multicast socket and how to set the TTL value for the packets.

    \section*{Socket programming exercises 3}
    This is an optional exercise on socket programming.
    Or, it can be assigned as a take-home project for extra credits.
    Note that familiarity with C programming is required.
    \subsection*{PROBLEM}
    Examine the message exchanges of FTP. Write a FTP client program which takes a file name as input, and upload the file to a standard FTP server on a remote machine.
    \subsection*{HINTS}
    \begin{itemize}
        \item First you need to set up the control connection to Port 21 of the remote machine, using a TCP socket.
        \item When the control connection is established, you need to exchange FTP commands with the remote FTP server, as given in Table 5.1.
        \item You can first run \textbf{telnet remote\_host 21}, then type \textbf{help} to list all the FTP commands.
        Also, you can try the commands out in the \textbf{telnet} window, e.g.\  use \textbf{USER guest} to send the user ID and \textbf{PASS guest1} to send the password to the FTP server.
        To terminate the \textbf{telnet} session, type \textbf{QUIT}.
        \item In your program, these messages should be sent to the FTP server by calling the \textbf{send()} function of the local TCP socket.
        \item Also your program needs to parse the server responses (some examples are given in Table 5.2) to find out the status of the previous FTP command.
        \item The FTP data connection should be established using the \textbf{PORT} command (see Chapter 5).
    \end{itemize}

\end{document}